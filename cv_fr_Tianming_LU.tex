%% start of file `template-zh.tex'.
%% Copyright 2006-2012 Xavier Danaux (xdanaux@gmail.com).
%
% This work may be distributed and/or modified under the
% conditions of the LaTeX Project Public License version 1.3c,
% available at http://www.latex-project.org/lppl/.

\documentclass[11pt,a4paper,sans]{mycv}   % possible options include font size ('10pt', '11pt' and '12pt'), paper size ('a4paper', 'letterpaper', 'a5paper', 'legalpaper', 'executivepaper' and 'landscape') and font family ('sans' and 'roman')

% moderncv 主题
\moderncvstyle{classic}                        % 选项参数是 ‘casual’, ‘classic’, ‘oldstyle’ 和 ’banking’
\moderncvcolor{blue}                          % 选项参数是 ‘blue’ (默认)、‘orange’、‘green’、‘red’、‘purple’ 和 ‘grey’
%\nopagenumbers{}                             % 消除注释以取消自动页码生成功能

% 字符编码
\usepackage[utf8]{inputenc}                   % 替换你正在使用的编码
%\usepackage{CJKutf8}

% 调整页面边距
\usepackage[scale={0.8, 0.95}]{geometry}
%\setlength{\hintscolumnwidth}{3cm}           % 如果你希望改变日期栏的宽度

% 个人信息
\name{Tianming}{LU}
\title{}
%\title{\small Maisel 1 Telecom ParisTech\newline Chambre 107\newline 212, rue de Tolbiac \newline 75013, Paris\newline Tel: 07 51 33 84 51\newline lutianming1005@gmail.com\newline Né le 05/10/1990\newline Nationalité: Chinois}                      % 可选项、如不需要可删除本行
\address{Maisel 1 Telecom ParisTech}{Chambre 107, 212, rue de Tobliac}{75013 Paris}             % 可选项、如不需要可删除本行
\phone[mobile]{07 51 33 84 51}                         % 可选项、如不需要可删除本行
\email{tianming.lu@telecom-paristech.fr}                    % 可选项、如不需要可删除本行
\social[github]{lutianming}
\birthday{Né le 05/10/1990}                  % 可选项、如不需要可删除本行
\nationality{Nationalité: Chinois}
\photo[64pt][0.4pt]{picture.eps}                  % ‘64pt’是图片必须压缩至的高度、‘0.4pt‘是图片边框的宽度 (如不需要可调节至0pt)、’picture‘ 是图片文件的名字;可选项、如不需要可删除本行
%\quote{引言(可选项)}                           % 可选项、如不需要可删除本行

% 显示索引号;仅用于在简历中使用了引言
%\makeatletter
%\renewcommand*{\bibliographyitemlabel}{\@biblabel{\arabic{enumiv}}}
%\makeatother

% 分类索引
%\usepackage{multibib}
%\newcites{book,misc}{{Books},{Others}}
%----------------------------------------------------------------------------------
%            内容
%----------------------------------------------------------------------------------
\begin{document}
%\begin{CJK}{UTF8}{gbsn}                       % 详情参阅CJK文件包
\makecvtitle

\section{Formation}
\cventry{2013 -- 2015}{Diplôme d'ingénieur}{Informatique}{Télécom ParisTech(ENST)}{en cours}{}
\cventry{2009 -- 2013}{Licence d'informatique}{Institut de
  Logiciels}{Université de Nankin}{Bac +4}{}  % 第3到第6编码可留白
\cventry{2009}{Examen d'admission à l'université}{équivalent du Baccalauréat}{}{}{}

\section{Expérience professionnelle}
\cventry{2012}{SDE Stage}{Microsoft}{Shangkin Chine}{}{Développeur de logiciels
\begin{itemize}%
\item construction d'un outil de recherche de tâche pour la FRS(Finances Reporting System) groupe:
  \begin{itemize}%
  \item avec SQL Server et C\#
  \item aide à comprendre le système
  \item aide à relancement les tâches lorsque les tâches spécifiques ont échoué
  \end{itemize}
\end{itemize}}

\section{Projets}
\cventry{2014}{Classification de texte avec SVM}{Python(Numpy, Scikit-learn, Pandas)}{}{}{
Télécom ParisTech, Paris
\begin{itemize}
\item la mise en œuvre de la classification de texte
\item comparaison des performances avec d'autres méthodes conventionnelles
\end{itemize}
}
\cventry{2014}{Création d'images à haute dynamique}{Python(Numpy, Python Image Library)}{}{}{
Télécom ParisTech, Paris
\begin{itemize}
\item Création des HDR images a partir des images originales
\end{itemize}
}
\cventry{2013-2014}{www.forumhorizonchine.com}{Python(Flask), HTML/CSS, JavaScript}{}{}{
L’Association amicale Franco Chinoise de Paristech (AFCP), Paris
  \begin{itemize}
  \item réécriture du site avec Flask
  \item ajout de nouvelles fonctions
  \end{itemize}
}

\cventry{2011}{Système de paiement mobile}{Java, Android}{}{}{
concoure de programmation par Citibank, Chine
\begin{itemize}
\item paiement rapide basé sur code à barres avec caméra mobile
\item communication entre le côté client et côté serveur
\end{itemize}
}

\cventry{2012}{Kinect classe}{C++(Qt)}{}{}{
un projet dans le séminaire de l'image et de l'analyse vidéo avec des informations de profondeur, Nankin Université, Chine
  \begin{itemize}
  \item aide aux enseignants avec Kinect
  \item implémentation de reconnaissance de gestes
  \end{itemize}
}

%\section{évaluation personnelle}
%\cventry{}{}{J'aime ce que j'apprends, et toujours obtenir
%   intéressé par des nouvelles choses. J'ai la capacité forte d'apprentissage  et l'adaptabilité. Je
%   préférer les tâches difficiles. Mais mon manque, c'est que je suis un peu introverti et pas très bon à traiter avec %des gens}{}{}{}

\section{Compétences spécifiques}
\cvitem{Langages}{Java, C/C++, Python, JavaScript, SQL, HTML/CSS, XML, Lisp}
\cvitem{Logiciels}{Linux, Emacs, Hadoop, Git, Eclipse, Visual Studio}
\cvitem{Statistiques}{Machine Learning, Data Mining}
\cvitem{Langues}{Chinois(langue maternelle), Anglais(courant), Français(bonnes notions)}
% \section{Langues}
% %\cvitemwithcomment{英语6级}{583}{}
% %\cvitemwithcomment{GRE}{319+4}{}
% \cvitemwithcomment{Anglais}{B2}{TOEFL 95}
% \cvitemwithcomment{Français}{B1}{en France depuis Juillet 2013}

% \section{计算机技能}
% \cvdoubleitem{类别 1}{XXX, YYY, ZZZ}{类别 4}{XXX, YYY, ZZZ}
% \cvdoubleitem{类别 2}{XXX, YYY, ZZZ}{类别 5}{XXX, YYY, ZZZ}
% \cvdoubleitem{类别 3}{XXX, YYY, ZZZ}{类别 6}{XXX, YYY, ZZZ}

\section{Loisirs}
\cvitem{}{les nouvelles technologies, natation, gymnastique, cinéma(dessin animé)}
% \section{其他 1}
% \cvlistitem{项目 1}
% \cvlistitem{项目 2}
% \cvlistitem{项目 3}

%\renewcommand{\listitemsymbol}{-}             % 改变列表符号

% \section{其他 2}
% \cvlistdoubleitem{项目 1}{项目 4}
% \cvlistdoubleitem{项目 2}{项目 5\cite{book1}}
% \cvlistdoubleitem{项目 3}{}

% 来自BibTeX文件但不使用multibib包的出版物
%\renewcommand*{\bibliographyitemlabel}{\@biblabel{\arabic{enumiv}}}% BibTeX的数字标签
%\nocite{*}
% \bibliographystyle{plain}
% \bibliography{publications}                    % 'publications' 是BibTeX文件的文件名

% 来自BibTeX文件并使用multibib包的出版物
%\section{出版物}
%\nocitebook{book1,book2}
%\bibliographystylebook{plain}
%\bibliographybook{publications}               % 'publications' 是BibTeX文件的文件名
%\nocitemisc{misc1,misc2,misc3}
%\bibliographystylemisc{plain}
%\bibliographymisc{publications}               % 'publications' 是BibTeX文件的文件名

\clearpage
%\end{CJK}

%-----       letter       ---------------------------------------------------------
% recipient data
% \recipient{Company Recruitment team}{Company, Inc.\\123 somestreet\\some city}
% \date{January 01, 1984}
% \opening{Dear Sir or Madam,}
% \closing{Yours faithfully,}
% \enclosure[Attached]{curriculum vit\ae{}}          % use an optional argument to use a string other than "Enclosure", or redefine \enclname
% \makelettertitle

% test

% \makeletterclosing

\end{document}



%% 文件结尾 `template-zh.tex'.