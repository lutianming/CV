%% start of file `template-zh.tex'.
%% Copyright 2006-2012 Xavier Danaux (xdanaux@gmail.com).
%
% This work may be distributed and/or modified under the
% conditions of the LaTeX Project Public License version 1.3c,
% available at http://www.latex-project.org/lppl/.


\documentclass[11pt,a4paper,sans]{moderncv}   % possible options include font size ('10pt', '11pt' and '12pt'), paper size ('a4paper', 'letterpaper', 'a5paper', 'legalpaper', 'executivepaper' and 'landscape') and font family ('sans' and 'roman')

% moderncv 主题
\moderncvstyle{classic}                        % 选项参数是 ‘casual’, ‘classic’, ‘oldstyle’ 和 ’banking’
\moderncvcolor{blue}                          % 选项参数是 ‘blue’ (默认)、‘orange’、‘green’、‘red’、‘purple’ 和 ‘grey’
%\nopagenumbers{}                             % 消除注释以取消自动页码生成功能

% 字符编码
\usepackage[utf8]{inputenc}                   % 替换你正在使用的编码
\usepackage{CJKutf8}

% 调整页面边距
\usepackage[scale=0.85]{geometry}
%\setlength{\hintscolumnwidth}{3cm}           % 如果你希望改变日期栏的宽度

% 个人信息
\firstname{陆}
\familyname{天明}
%\title{简历}                      % 可选项、如不需要可删除本行
\address{生日 1990-10-05}{江苏 太仓}{}             % 可选项、如不需要可删除本行
\mobile{189 1306 8026}                         % 可选项、如不需要可删除本行
\email{tianming.lu@foxmail.com}                  %可选项、如不需要可删除本
                                %% 行
\social[github]{lutianming}
\photo[64pt][0.4pt]{picture.jpg}                  % ‘64pt’是图片必须压缩至的高度、‘0.4pt‘是图片边框的宽度 (如不需要可调节至0pt)、’picture‘ 是图片文件的名字;可选项、如不需要可删除本行
%\quote{引言(可选项)}                           % 可选项、如不需要可删除本行

% 显示索引号;仅用于在简历中使用了引言
%\makeatletter
%\renewcommand*{\bibliographyitemlabel}{\@biblabel{\arabic{enumiv}}}
%\makeatother

% 分类索引
%\usepackage{multibib}
%\newcites{book,misc}{{Books},{Others}}
%----------------------------------------------------------------------------------
%            内容
%----------------------------------------------------------------------------------
\begin{document}
\begin{CJK}{UTF8}{gbsn}                       % 详情参阅CJK文件包
\maketitle

\section{教育背景}
\cventry{2013--2015}{Diplôme d'ingénieur(硕士)}{Computer Science}{Telecom
  ParisTech 巴黎高科}{法国}{
  \begin{itemize}
  \item 连续2年获得企业奖学金
  \end{itemize}  
  }
\cventry{2009--2013}{本科}{软件工程}{南京大学}{}{}  % 第3到第6编码可留白

%\section{毕业论文}
%\cvitem{题目}{\emph{题目}}
%\cvitem{导师}{导师}
%\cvitem{说明}{\small 论文简介}

% \section{工作背景}
% \subsection{专业}
% \cventry{年 -- 年}{职位}{公司}{城市}{}{不超过1--2行的概况说明\newline{}%
% 工作内容:%
% \begin{itemize}%
% \item 工作内容 1;
% \item 工作内容 2、 含二级内容:
%   \begin{itemize}%
%   \item 二级内容 (a);
%   \item 二级内容 (b)、含三级内容 (不建议使用);
%     \begin{itemize}
%     \item 三级内容 i;
%     \item 三级内容 ii;
%     \item 三级内容 iii;
%     \end{itemize}
%   \item 二级内容 (c);
%   \end{itemize}
% \item 工作内容 3。
% \end{itemize}}
% \cventry{年 -- 年}{职位}{公司}{城市}{}{说明行1\newline{}说明行2}
% \subsection{其他}
% \cventry{年 -- 年}{职位}{公司}{城市}{}{说明}
\section{工作背景}
\cventry{2020/06 -- 至今}{资深研发工程师}{SmartNews}{上海}{}{%
\begin{itemize}%
\item 设计实现基于Flink的实时处理平台,为公司业务的实时化迁移提供平台基础
  \begin{itemize}%
  \item 设计实现Flink任务的统一提交、管控。提供可视化、多租户等功能,降低批处理任务的迁移门槛
  \item 负责实时平台与亚马逊AWS、Datadog等第三方服务的整合工作
  \end{itemize}
\end{itemize}}
\cventry{2019/03 --2020/06}{研发经理}{星环科技}{上海}{}{%
\begin{itemize}%
\item 负责AI平台多个服务的设计、开发、管理工作
  \begin{itemize}%
  \item 设计实现微服务框架,推动产品的微服务化改造,打通产品间的交互,使得整个平台更成体系
  \item 负责推动从平台产品到解决方案的演进
  \item 负责前后端研发资源的分配、协调和日常管理工作
  \end{itemize}
\end{itemize}}
\cventry{2016/03 -- 2019/03}{高级研发工程师}{星环科技}{上海}{}{%
\begin{itemize}%
\item 负责机器学习平台产品设计和研发,使用Spark, Spring, Tensorflow, Kubernetes等相关技术,从无到有完成新产品的设计开发,不断迭代
  \begin{itemize}%
  \item 实现拖拽式构建机器学习任务的功能。通过算子逻辑组合构建流程DAG图,后端解析DAG图生产实际执行计划执行对应的机器学习流程
  \item 定制化Livy实现多用户多任务的Spark执行环境。根据平台需求,修改和扩展Livy的功能,包括资源控制、任务历史记录等等
  \item 实现Spark与Tensorflow的整合。同一流程DAG图混合Spark SQL逻辑和深度学习逻辑,充分发挥Spark的大数据预处理能力和Tensorflow的深度学习建模能力。
  \item 优化Spark模型部署运行速度。封装Spark SQL接口,实现对应的无Spark运行环境的接口,保证线上预测功能的性能。
        模型预测REST API响应时间从秒级下降到毫秒级  
  \item 设计实现基于Docker和Kubernetes的模型版本管理和上线等功能。实现基于Dockerfile的模型CICD功能,从模型文件构建模型镜像。以镜像形式完成跨集群部署功能。
  \end{itemize}
\end{itemize}}


\section{专业技能}
%\cvitem{编程语言}{Scala, Java, Python}
%\cvitem{大数据}{Spark, Kafka, HDFS, Hadoop}
%\cvitem{后端}{Spring, Redis, MySQL}
%\cvitem{云}{}
%\cvitem{工具}{}
%\cvitem{统计}{}
%\cvitem{语言}{}
\cvlistdoubleitem{编程语言:Scala, Java, Python}{大数据:Spark, Kafka, HDFS, Hadoop}
\cvlistdoubleitem{后端:Spring, Redis, MySQL}{云:Docker, Kubernetes}
\cvlistdoubleitem{工具:Idea, Git, Linux}{统计:Machine Learning, Data Mining}
\cvlistdoubleitem{语言:汉语(母语), 英语, 法语}{}

% \section{计算机技能}
% \cvdoubleitem{类别 1}{XXX, YYY, ZZZ}{类别 4}{XXX, YYY, ZZZ}
% \cvdoubleitem{类别 2}{XXX, YYY, ZZZ}{类别 5}{XXX, YYY, ZZZ}
% \cvdoubleitem{类别 3}{XXX, YYY, ZZZ}{类别 6}{XXX, YYY, ZZZ}


% \section{其他 1}
% \cvlistitem{项目 1}
% \cvlistitem{项目 2}
% \cvlistitem{项目 3}

%\renewcommand{\listitemsymbol}{-}             % 改变列表符号

% \section{其他 2}
% \cvlistdoubleitem{项目 1}{项目 4}
% \cvlistdoubleitem{项目 2}{项目 5\cite{book1}}
% \cvlistdoubleitem{项目 3}{}

% 来自BibTeX文件但不使用multibib包的出版物
%\renewcommand*{\bibliographyitemlabel}{\@biblabel{\arabic{enumiv}}}% BibTeX的数字标签
%\nocite{*}
%\bibliographystyle{plain}
%\bibliography{publications}                    % 'publications' 是BibTeX文件的文件名

% 来自BibTeX文件并使用multibib包的出版物
%\section{出版物}
%\nocitebook{book1,book2}
%\bibliographystylebook{plain}
%\bibliographybook{publications}               % 'publications' 是BibTeX文件的文件名
%\nocitemisc{misc1,misc2,misc3}
%\bibliographystylemisc{plain}
%\bibliographymisc{publications}               % 'publications' 是BibTeX文件的文件名

\clearpage
\end{CJK}
\end{document}


%% 文件结尾 `template-zh.tex'.

%%% Local Variables:
%%% mode: latex
%%% TeX-master: t
%%% End:
