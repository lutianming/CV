\documentclass[a4paper]{article}
\title{Statement of Purpose}
\author{Tianming Lu \\
		Software Institute, Nanjing University\\
		lutianming1005@gmail.com}
\date{}
\begin{document}
\maketitle
\paragraph{}
I can still remember that in the first class of Software Engineering, our professor told us that software is one of the most complex systems in the world and our duty is to make functional, powerful and reliable software that can make people's life better. In the first two years of university study, I have gained lots of essential knowledge in computer science and software engineering, such as data structure, algorithm and design pattern. Gradually, I began to show interest in finding information in images. As a result, in the third year, I chose Multimedia as my main direction. And now I narrow my interests down to image process and machine learning.
And the more I know in this field, the stronger I feel that the knowledge in my undergraduate period is not enough at all. Thus, I plan to further my education. 
\paragraph{}
I have already learned that French engineer education is rather efficient for training an elite in engineering. Students not only learn in class, but also learn in company. The theory and practice are highly combined with each other, which is proved to be an efficient way to learn and teach. What is more, With the support of CTI, engineer education in France can achieve effective quality assurance. As a result, being a student whose speciality is engineering, I believe France is a right choice for me to further my education. And ParisTech, with each school recognized as the best in its domain in France, is the place I should go without doubt. Especially, I find that Telecom ParisTech attracts me the most because of Signal \& Image Process Department within it. The curriculum provided by the TSI department can meet my professional goals perfectly.
\paragraph{}
My professional goals are extremely clear. I plan to be a software engineering in image processing or machine learning fields, which I am interested in, after finishing my study. To reach my goal, when I was in third grade, I asked to participate the Seminar of Kinect and Depth Image, which was hold for graduate students in my institute. In the seminar, we read papers, made presentations and did research project. It was a big challenge for me. And during the seminar, I implemented a simple gesture recognition system. It was full of fun. I found there are four research groups in STI, telecom ParisTech. They all hold many interesting seminars. For example, Statistical Machine Learning in Paris hold by STA group. I would like to be a kind of engineer that can find solutions to help companies make more profit and help employees save time. Three months experience of internship in Microsoft, China help me not only gain lots of practical skills, but also understand this industry. The department I worked in is the most burdensome one, and my mentor nearly have lots of work to do all the time. And my job in the internship period is to build tools to help them finish job faster and easier. When I see my job can make others more efficient and relaxed, I feel a sense of accomplishment. That is why I want be an engineer solving real problems.
\paragraph{}   
During the past several years of study, I have built a strong base for my future career. I know how software works and how to make it more reliable and friendly.  But I still feel that I do not have enough knowledge in this field. The curriculum provided by the TSI department includes image, multimedia, machine learning, data mining and so on. All these courses perfectly satisfy my professional goals. It can help me fill the gap between general knowledge and specialized knowledge. This is what ParisTech can help me.
\end{document}