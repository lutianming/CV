%%author: Tianming Lu
%%date: 2012-4-29



\documentclass[11pt,a4paper,sans]{moderncv}   % possible options include font size ('10pt', '11pt' and '12pt'), paper size ('a4paper', 'letterpaper', 'a5paper', 'legalpaper', 'executivepaper' and 'landscape') and font family ('sans' and 'roman')

% moderncv themes
\moderncvstyle{classic}                        % style options are 'casual' (default), 'classic', 'oldstyle' and 'banking'
\moderncvcolor{black}                          % color options 'blue' (default), 'orange', 'green', 'red', 'purple', 'grey' and 'black'
%\renewcommand{\familydefault}{\sfdefault}    % to set the default font; use '\sfdefault' for the default sans serif font, '\rmdefault' for the default roman one, or any tex font name
%\nopagenumbers{}                             % uncomment to suppress automatic page numbering for CVs longer than one page

% character encoding
%\usepackage[utf8]{inputenc}                  % if you are not using xelatex ou lualatex, replace by the encoding you are using
%\usepackage{CJKutf8}                         % if you need to use CJK to typeset your resume in Chinese, Japanese or Korean

% adjust the page margins
\usepackage[scale=0.8]{geometry}
%\setlength{\hintscolumnwidth}{3cm}           % if you want to change the width of the column with the dates
%\setlength{\maketitlenamewidth}{10cm}        % for the 'classic' style, if you want to force the width allocated to your name and avoid line breaks. be careful though, the length is normally calculated to avoid any overlap with your personal info; use this at your own typographical risks...

% personal data
\firstname{Tianming}
\familyname{Lu}
\title{CV}               % optional, remove the line if not wanted
\address{GuLou Campus of Nanjing University}{210093 Nanjing}    % optional, remove the line if not wanted
\mobile{0086-15298387084}                     % optional, remove the line if not wanted
\email{lutianming1005@gmail.com}                          % optional, remove the line if not wanted
%\extrainfo{additional information}            % optional, remove the line if not wanted
\photo[64pt][0.4pt]{picture.jpg}                  % '64pt' is the height the picture must be resized to, 0.4pt is the thickness of the frame around it (put it to 0pt for no frame) and 'picture' is the name of the picture file; optional, remove the line if not wanted
%\quote{Some quote (optional)}                 % optional, remove the line if not wanted

% to show numerical labels in the bibliography (default is to show no labels); only useful if you make citations in your resume
%\makeatletter
%\renewcommand*{\bibliographyitemlabel}{\@biblabel{\arabic{enumiv}}}
%\makeatother

% bibliography with mutiple entries
%\usepackage{multibib}
%\newcites{book,misc}{{Books},{Others}}
%----------------------------------------------------------------------------------
%            content
%----------------------------------------------------------------------------------
\begin{document}
%\begin{CJK*}{UTF8}{gbsn}                     % to typeset your resume in Chinese using CJK
%-----       resume       ---------------------------------------------------------
\makecvtitle

\section{Education}
\cventry{2009--2012}{Bachelor}{Software Engineering}{Nanjing
  University}{}{Overall GPA: 86, Rank: 12/185}  % arguments 3 to 6 can be left empty

% \section{Master thesis}
% \cvitem{title}{\emph{Title}}
% \cvitem{supervisors}{Supervisors}
% \cvitem{description}{Short thesis abstract}

% \section{Experience}
% \subsection{Vocational}
% \cventry{year--year}{Job title}{Employer}{City}{}{General description no longer than 1--2 lines.\newline{}%
% Detailed achievements:%
% \begin{itemize}%
% \item Achievement 1;
% \item Achievement 2, with sub-achievements:
%   \begin{itemize}%
%   \item Sub-achievement (a);
%   \item Sub-achievement (b), with sub-sub-achievements (don't do this!);
%     \begin{itemize}
%     \item Sub-sub-achievement i;
%     \item Sub-sub-achievement ii;
%     \item Sub-sub-achievement iii;
%     \end{itemize}
%   \item Sub-achievement (c);
%   \end{itemize}
% \item Achievement 3.
% \end{itemize}}
% \cventry{year--year}{Job title}{Employer}{City}{}{Description line 1\newline{}Description line 2}
% \subsection{Miscellaneous}
% \cventry{year--year}{Job title}{Employer}{City}{}{Description}

\section{Experience}
\cventry{2012/7--2012/9}{SDE Intern}{Microsoft}{Shanghai, China}{}{Software development engineer intern\newline{}%
Detailed achievements:%
\begin{itemize}%
\item build a job query tool for the FRS(Finance Reporting System) group:
  \begin{itemize}%
  \item help developers understand the system;
  \item help rerun jobs when jobs fail;
  \end{itemize}
\end{itemize}}

\section{Projects}
\cventry{2011/7-2011/8}{Ipay mobile payment system}{}{entry for a competition hold by Citibank}{}{quick payment with
  barcode from mobile's camera\newline{}my work: data parsing}
\cventry{2011/11-2012/3}{Kinect Classroom}{}{a project in the seminar of Image and Video Analysis with Depth Information}{}{use Kinect to help teachers\newline{}my work: gesture recognition}

\section{Languages}
\cvitemwithcomment{GRE}{319+4}{}
\cvitemwithcomment{TOEFL}{95}{}

\section{Interests}
\cvitem{computer vision}{}
\cvitem{machine learning}{online machine learning class}
\cvitem{programming languages}{Python, Lisp}{}
\cvitem{swimming}{}

% \section{Extra 1}
% \cvlistitem{Item 1}
% \cvlistitem{Item 2}
% \cvlistitem{Item 3}

% \renewcommand{\listitemsymbol}{-~}            % change the symbol for lists

% \section{Extra 2}
% \cvlistdoubleitem{Item 1}{Item 4}
% \cvlistdoubleitem{Item 2}{Item 5\cite{book1}}
% \cvlistdoubleitem{Item 3}{}

% Publications from a BibTeX file without multibib\renewcommand*{\bibliographyitemlabel}{\@biblabel{\arabic{enumiv}}}% for BibTeX numerical labels
\nocite{*}
\bibliographystyle{plain}
\bibliography{publications}                   % 'publications' is the name of a BibTeX file

% Publications from a BibTeX file using the multibib package
%\section{Publications}
%\nocitebook{book1,book2}
%\bibliographystylebook{plain}
%\bibliographybook{publications}              % 'publications' is the name of a BibTeX file
%\nocitemisc{misc1,misc2,misc3}
%\bibliographystylemisc{plain}
%\bibliographymisc{publications}              % 'publications' is the name of a BibTeX file

\clearpage

%\clearpage\end{CJK*}                         % if you are typesetting your resume in Chinese using CJK; the \clearpage is required for fancyhdr to work correctly with CJK, though it kills the page numbering by making \lastpage undefined
\end{document}


%% end of file `template.tex'.
